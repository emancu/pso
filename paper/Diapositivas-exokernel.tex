\documentclass[10pt]{beamer}
\usepackage[spanish]{babel}
\usepackage[utf8]{inputenc}
\usepackage{color}
\usepackage{framed}
\usepackage{endnotes}
\usepackage{graphicx}
\usepackage{multicol}

\usetheme{Hannover}

\title{Exokernel}
\subtitle{Clase práctica AED II}

\author{E. Mancuso\\ A. Mataloni\\ M. Miguel }

\date{16 de Junio de 2011}

\begin{document}
%-----slide------------------init--------------------
 \begin{frame}
  \titlepage
 \end{frame}
 \begin{frame}
  \tableofcontents
 \end{frame}
%----------------------------------------------------
\section{Introducción}
%-----slide------------------intro-------------------
\begin{frame}{Introducción}
Un exokernel 

\begin{itemize}
 \item Agrupar información
 \item Simplificar TADs mediante el uso de otros y sus propiedades
\end{itemize}

En resumen, hacer que el TAD sea lo más simple y conciso posible, expresando todo lo que debe expresar. 
\end{frame}
%----------------------------------------------------
\end{document}